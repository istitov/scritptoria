\documentclass[12pt]{scrartcl}
\usepackage[utf8]{inputenc}
\usepackage{amsmath}
\usepackage{pstricks-add}
\pagestyle{empty}

\usepackage{pstricks}
\usepackage{pst-circ}
\usepackage{pst-plot}
\begin{document}
\begin{figure}[ht]
\begin{pspicture}(-1.5,-1)(15,8.5)
%[showgrid=true,subgriddiv=1,griddots=10]
% Node definitions
\pnode(0,0){A}
\pnode(1,0){B}
\pnode(3,0){C}
\pnode(6,0){D}
\pnode(9,0){E}
\pnode(12,0){F}
\pnode(14,0){G}

\pnode(0,7){H}
\pnode(1,7){I}
\pnode(3,7){J}
\pnode(6,7){K}
\pnode(9,7){L}
\pnode(12,7){M}

\pnode(6,3.5){N}
\pnode(9,3.5){O}

% Dipole node connection

\tension[labeloffset=-0.5](A)(H){$U_{gen}$}

\resistor[labeloffset=-1,tensionlabeloffset=1.5,tensionlabel=$U_A$](D)(N){$R_A$}
\resistor[labeloffset=-1,tensionlabeloffset=1.5,tensionlabel=$U_S$](N)(K){$R_S$}

\circledipole[labeloffset=0](E)(O){\Large\textbf{V}}
\uput[ur](9.5,1.5){$U_A$}

\circledipole[labeloffset=0](F)(M){\Large\textbf{V}}
 \uput[ur](12.5,3.5){$U_{ges,m}$}

\resistor[labeloffset=-0.6](I)(J){$50 \Omega$}

\psframe[linestyle=dashed,dash=3pt 2pt](5.5,4.5)(6.5,6)
\psframe[linestyle=dashed,dash=3pt 2pt](-1,-1)(3,8)
\psframe[linestyle=dashed,dash=3pt 2pt](3.5,-1)(7.5,8)
\psframe[linestyle=dashed,dash=3pt 2pt](8,-1)(14.5,8)

\uput[0](-1,0){$GND$}
\uput[0](-1,7){$AO0+$}
\uput[ur](8,7){$AI0$}

% Wire to complete circuit
\wire(B)(G)
\wire(J)(K)
\wire(K)(L)
\wire(L)(M)
\wire(N)(O)
% Ground
\newground[groundstyle=old](G)
\end{pspicture}
\caption{Two-point method.}
\end{figure}

\begin{figure}[ht]
\begin{pspicture}(-1.5,-1)(16,12)
%[showgrid=true,subgriddiv=1,griddots=10]
%Node definitions
\pnode(0,0){A}
\pnode(1,0){B}
\pnode(3,0){C}
\pnode(6,0){D}
\pnode(9,0){E}
\pnode(13,0){F}
\pnode(14,0){G}

\pnode(0,10.5){H}
\pnode(1,10.5){I}
\pnode(3,10.5){J}
\pnode(6,10.5){K}
\pnode(9,10.5){L}
\pnode(13,10.5){M}

\pnode(6,7.5){N}
\pnode(6,8){N1}
\pnode(11,8){N2}
\pnode(11,0){N3}

\pnode(9,7.5){O}
\pnode(1,3.5){P}
\pnode(3,3.5){Q}
\pnode(6,3.5){R}


% Dipole node connection

\tension[labeloffset=-0.7](A)(H){$U_{gen1}$}
\tension[labeloffset=-0.7](B)(P){$U_{gen2}$}

\resistor[labeloffset=-1,tensionlabeloffset=1.5,tensionlabel=$U_A$](R)(N){$R_A$}
\resistor[labeloffset=-1,tensionlabeloffset=1.5,tensionlabel=$U_S$](N)(K){$R_S$}

\circledipole[labeloffset=0](E)(O){\Large\textbf{V}}
\uput[ur](9.5,3.5){$U_A$}

\circledipole[labeloffset=0](F)(M){\Large\textbf{V}}
\uput[ur](11.5,4){$U_{ges,m1}$}

\circledipole[labeloffset=0](N2)(N3){\Large\textbf{V}}
\uput[ur](13.5,5){$U_{ges,m2}$}

\resistor[labeloffset=-0.6](I)(J){$50 \Omega$}
\resistor[labeloffset=-0.6](P)(Q){$50 \Omega$}

\psframe[linestyle=dashed,dash=3pt 2pt](5.5,8.2)(6.5,9.7)
\psframe[linestyle=dashed,dash=3pt 2pt](-1,-1)(3,11.5)
\psframe[linestyle=dashed,dash=3pt 2pt](3.5,3)(7.5,11.5)
\psframe[linestyle=dashed,dash=3pt 2pt](8,-1)(15.5,11.5)

\uput[0](-1,0){$GND$}
\uput[0](-1,10.5){$AO0+$}
\uput[u](1,4){$AO1+$}

\uput[dr](8,7.5){$AI0$}
\uput[dr](8,8.5){$AI1$}
\uput[dr](8,10.5){$AI2$}

% Wire to complete circuit
\wire(B)(G)
\wire(J)(K)
\wire(K)(L)
\wire(L)(M)
\wire(N)(O)
\wire(Q)(R)
\wire(N1)(N2)
% Ground
\newground[groundstyle=old](G)
\end{pspicture}
\caption{Four-point method.}
\end{figure}
\begin{figure}
\begin{pspicture}[showgrid=true,subgriddiv=1,griddots=10]
(-1,-1)(12,9.5)

\logicic[nicpins=32,%
pintl=false,pintllabel=tl,pintlnumber=1,%
pintc=false,pintclabel=tc,pintcnumber=2,%
pintr=false,pintrlabel=tr,pintrnumber=3,%
invertpintl=true,invertpintc=true,invertpintr=true,%
pinbl=false,pinbllabel=bl,pinblnumber=1,%
pinbc=false,pinbclabel=bc,pinbcnumber=2,%
pinbr=false,pinbrlabel=br,pinbrnumber=3,%
invertpinbl=true,invertpinbc=true,invertpinbr=true,%
pinqlabel=GND,pinrlabel=AI0,pinslabel=AI4,pintlabel=GND,%
pinulabel=AI1,pinvlabel=AI5,pinwlabel=GND,pinxlabel=AI2,%
pinylabel=AI6,pinzlabel=GND,pinaalabel=AI3,pinablabel=AI7,%
pinaclabel=GND,pinadlabel=AO0,pinaelabel=AO1,pinaflabel=GND,
pinqnumber=1,pinrnumber=2,pinsnumber=3,pintnumber=4,%
pinunumber=5,pinvnumber=6,pinwnumber=7,pinxnumber=8,%
pinynumber=9,pinznumber=10,pinaanumber=11,pinabnumber=12,%
pinacnumber=13,pinadnumber=14,pinaenumber=15,pinafnumber=16,
invertpinq=true,invertpinr=true,invertpins=true,invertpint=true,%
invertpinu=true,invertpinv=true,invertpinw=true,invertpinx=true,%
invertpiny=true,invertpinz=true,invertpinaa=true,invertpinab=true,%
invertpinac=true,invertpinad=true,invertpinae=true,invertpinaf=true]
(0,0){NI USB-6009}

\psframe[linestyle=dashed,dash=3pt 2pt](6.8,0)(8.2,8.5)
\psframe[linestyle=dashed,dash=3pt 2pt](9.5,0)(11.5,8.5)

\pnode(5,1.0){A}
\pnode(5,2.5){B}
\pnode(5,4.0){C}
\pnode(5,7.0){D}
\pnode(5,7.5){E}

%\pnode(7,1.0){A}
%\pnode(7,2.5){B}
%\pnode(7,4.0){C}
%\pnode(7,7.0){D}
%\pnode(7,7.5){E}

\pnode(7.5,1){F}
\pnode(7.5,3){G}
\pnode(10,6.5){H}
\pnode(10,3.0){K}
\pnode(11,7.0){I}
\pnode(11,3.0){J}
\resistor[labeloffset=0,labelangle=90](F)(G){$R_A$}
\wire(A)(F)
\wire(E)(G)

\transformer[dipolestyle=rectangle](H)(K)(I)(J){$R_S$}
\wire(F)(J)
\wire(D)(I)
\wire(C)(H)
\wire(B)(K)

\end{pspicture}
\end{figure}
\end{document}
